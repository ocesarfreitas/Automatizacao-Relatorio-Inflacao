\documentclass[xcolor={dvipsnames}]{beamer}
\usetheme{Copenhagen}

\definecolor{upforestgreen}{rgb}{0.0, 0.27, 0.13}
\definecolor{coolblack}{rgb}{0.0, 0.18, 0.39}

\setbeamercolor*{structure}{bg=upforestgreen!20,fg=upforestgreen}
\setbeamercolor*{palette primary}{use=structure,fg=white,bg=structure.fg}
\setbeamercolor*{palette secondary}{use=structure,fg=white,bg=structure.fg}
\setbeamercolor*{palette tertiary}{use=structure,fg=white,bg=structure.fg}
\setbeamercolor*{palette quaternary}{fg=white,bg=structure.fg!5!coolblack}
\setbeamercolor{titlelike}{parent=palette primary,fg=white}
\setbeamercolor{frametitle}{bg=gray!10!white,fg=structure.fg!75!white}
\setbeamercolor{sidebar}{fg=white,bg=white}

\usepackage{graphicx, booktabs, subcaption, textpos, palatino, hyperref, amsmath, smartdiagram}
\usepackage[export]{adjustbox}
\usepackage[portuguese]{babel}
\setbeamercovered{highly dynamic}
\newcounter{saveenumi}
\newcommand{\seti}{\setcounter{saveenumi}{\value{enumi}}}
\newcommand{\conti}{\setcounter{enumi}{\value{saveenumi}}}

\title[Mulheres no Sudeste]{Mulheres no Sudeste: Uma análise salarial}
\subtitle{PNAD Contínua 2012}
\author[Albuquerque \and Linhares \and Batista]{César Freitas Albuquerque \and Júlia Michalski Linhares \and Sarah Moreira Batista}
\institute[UnB]{UnB - Universidade de Brasília}
\date{28 de outubro de 2021}

\begin{document}
\begin{frame}
    \titlepage
    \begin{figure}[htpb]
            \includegraphics[width=0.225\linewidth, left]{Figuras/download.png}
    \end{figure}
\end{frame}

\addtobeamertemplate{frametitle}{}{%
\begin{textblock*}{75mm}(0.9\textwidth,-0.9cm)
\includegraphics[height=0.8cm,width=1.7cm]{Figuras/download.png}
\end{textblock*}}

\begin{frame}
    \tableofcontents
\end{frame}

\section{Amostra e Análises Descritivas}
\begin{frame}
    \tableofcontents[currentsection]
\end{frame}
\subsection{Amostra}
\begin{frame}{Amostra}
    \begin{itemize}
        \item Base de dados: PNAD Contínua 2012
        \item 293.858 indivíduos para compor a amostra da região Sudeste
        \item Variável \textit{V1028} para os pesos
        \item Variável \textit{cor}: desconsideramos as etnias asiáticas e indígenas
    \end{itemize}
\end{frame}

\begin{frame}{Amostra}
    \begin{table}[H]
\caption{Distribuição da Amostra para a região Sudeste}
\centering
\begin{adjustbox}{width=.8\columnwidth,center}
\begin{tabular}{@{}llccccc@{}}
\toprule
\textbf{Gênero} &
  \textbf{Cor} &
  \multicolumn{1}{l}{\textbf{ES}} &
  \multicolumn{1}{l}{\textbf{MG}} &
  \multicolumn{1}{l}{\textbf{RJ}} &
  \multicolumn{1}{l}{\textbf{SP}} &
  \multicolumn{1}{l}{\textbf{Total}} \\ \midrule
\textbf{Homem}  & Branco   & 17702 & 8624  & 16505 & 27121 & 69952  \\
\textbf{}       & Preto    & 3715  & 1479  & 5231  & 2489  & 12914  \\
\textbf{}       & Amarelo  & 72    & 35    & 67    & 519   & 693    \\
\textbf{}       & Pardo    & 19839 & 9742  & 15736 & 13086 & 58403  \\
\textbf{}       & Indígena & 34    & 15    & 49    & 65    & 163    \\
\textbf{Mulher} & Branco   & 18572 & 9555  & 19624 & 29599 & 77350  \\
\textbf{}       & Preto    & 3661  & 1449  & 5592  & 2537  & 13239  \\
\textbf{}       & Amarelo  & 60    & 32    & 71    & 561   & 724    \\
\textbf{}       & Pardo    & 20171 & 9843  & 16806 & 13402 & 60222  \\
\textbf{}       & Indígena & 32    & 29    & 62    & 75    & 198    \\ \midrule
\textbf{Total}  &          & 83858 & 40803 & 79743 & 89454 & 293858 \\ \bottomrule
\end{tabular}
\end{adjustbox}
\caption*{\textbf{Fonte:} PNAD Contínua - IBGE (2012)}
\end{table}
\end{frame}

\begin{frame}{Situação no domicílio}
    \begin{table}[h]
\begin{adjustbox}{width=\columnwidth,center}
\begin{tabular}{lcccccc}
\hline
\textbf{Estado} & \textbf{Situação do Domicílio} & \textbf{Porcentagem} & \textbf{Pessoas/Domicílio}\\ \hline
\textbf{MG} & \textbf{Urbano} & 72,32\% & 3.80 \\ 
\textbf{}   & \textbf{Rural}  & 27,68\% & 3.94 \\\hline
\textbf{ES} & \textbf{Urbano} & 80,72\% & 3,73 \\ 
\textbf{}   & \textbf{Rural}  & 19,28\% & 3,83 \\\hline
\textbf{RJ} & \textbf{Urbano} & 92,75\% & 3,64 \\
\textbf{}   & \textbf{Rural}  & 7,25\%  & 4,03 \\ \hline
\textbf{SP} & \textbf{Urbano} & 83,23\% & 3,79 \\
\textbf{}   & \textbf{Rural}  & 16,77\% & 1,58 \\\hline
\end{tabular}
\end{adjustbox}
\end{table}

\end{frame}

\subsection{Renda}
\begin{frame}{Desigualdade de renda}
\begin{figure}
\centering
  \includegraphics[width=\linewidth]{Figuras/trabalho mulher 02 (1).png}
\end{figure}
\end{frame}

\begin{frame}{Decis de renda}
    \begin{table}[!h]
    \caption{Decis de renda da população total e por gênero (em R\$)\\
    Região Sudeste}
    \begin{adjustbox}{width=0.6\columnwidth,center}
    \begin{tabular}{lccc} \hline
    \textbf{Decis} & \textbf{Renda Total} & \textbf{Renda Homem} & \textbf{Renda Mulher}\\\hline
    10\%           & 500,00     & 600,00     & 400,00    \\
    20\%           & 622,00     & 700,00     & 600,00    \\
    30\%           & 700,00     & 800,00     & 622,00    \\
    40\%           & 800,00     & 950,00     & 700,00    \\
    50\%           & 1.000,00   & 1.100,00   & 800,00    \\
    60\%           & 1.200,00   & 1.250,00   & 900,00    \\
    70\%           & 1.500,00   & 1.500,00   & 1.200,00  \\
    80\%           & 2.000,00   & 2.000,00   & 1.500,00  \\
    90\%           & 3.000,00   & 3.500,00   & 2.500,00  \\\hline
    99\%*          & 12.000,00  & 14.514,69  & 9.000,00  \\ 
    99,9\%*        & 25.000,00  & 30.000,00  & 18.000,00 \\\hline
    \textbf{Mínimo}& 1,00       & 1,00       & 1,00      \\ 
    \textbf{Máximo}& 120.000,00 & 120.000,00 & 50.000,00 \\
    \textbf{Média} & 1.441,00   & 1663,00    & 1.172,00  \\\hline
    \end{tabular}
    \end{adjustbox}
    \caption*{\textbf{Fonte:} PNAD Contínua IBGE (2012)}
\end{table}
\end{frame}

\begin{frame}{Indíce de Gini}
    \begin{table}[h]
    \caption{Índice de Gini por UF e gênero}
    \centering
    \begin{tabular}{lcccc} \hline
    \textbf{Estados} & Homem  & Mulher & Total  \\\hline
    Espírito Santo   & 0,4780 & 0,4447 & 0,4734 \\
    Minas Gerais     & 0,4687 & 0,4412 & 0,4686 \\
    Rio de Janeiro   & 0,4668 & 0,4555 & 0,4685 \\
    São Paulo        & 0,4887 & 0,4610 & 0,4864 \\\hline
    Total            & 0,4874 & 0,4612 & 0,4849 \\\hline
    \end{tabular}
    \caption*{\textbf{Fonte:} PNAD Contínua IBGE (2012)}
\end{table}

\end{frame}

\begin{frame}{Diferença entre horas por gênero}
    \begin{figure}
        \centering
        \includegraphics[width=.8\linewidth]{Figuras/trabalho mulher 06 (1).png}
        \label{fig:my_label}
    \end{figure}
\end{frame}

\subsection{Mercado de Trabalho}
\begin{frame}{Taxa de ocupação}
    \begin{table}[H]
    \caption{Taxa de ocupação por estado (\%)}
    \centering
    \begin{tabular}{lccc}
    \textbf{Estados} & Homem & Mulher& Total \\\hline
    Espírito Santo   & 94,48 & 90,56 & 92,77 \\
    Minas Gerais     & 94,23 & 91,29 & 92,97 \\
    Rio de Janeiro   & 93,58 & 90,16 & 92,03 \\
    São Paulo        & 94,00 & 90,85 & 92,59 \\\hline
    Total            & 94,05 & 90,67 & 92,54 \\\hline         
    \end{tabular}
    \caption*{\textbf{Fonte:} PNAD Contínua IBGE (2012)}
\end{table}

\end{frame}
\begin{frame}{Ocupações principais por gênero}
    \begin{figure}
        \centering
        \includegraphics[width=1.1\linewidth]{Figuras/trabalho mulher 09.png}
    \end{figure}
\end{frame}

\begin{frame}{Motivo pelo qual não procurou emprego}
    \begin{figure}
        \centering
        \includegraphics[width=\linewidth]{Figuras/Motivo_Desemprego.png}
    \end{figure}
\end{frame}

\subsection{Educação}
\begin{frame}{Nível de educação por gênero}
    \begin{figure}
        \centering
        \includegraphics[width=\linewidth]{Figuras/educação 01.png}
    \end{figure}
\end{frame}

\begin{frame}{Média de renda por nível educacional}
    \begin{figure}
        \centering
        \includegraphics[width=\linewidth]{Figuras/educação 03.png}
    \end{figure}
\end{frame}

\begin{frame}{Prop. das faixas salariais por nível educacional}
    \begin{figure}
        \centering
        \includegraphics[width=1.1\linewidth]{Figuras/educação 04.png}
    \end{figure}
\end{frame}

\begin{frame}{Nem-nem}
    \begin{figure}
        \centering
        \includegraphics[width=.75\linewidth]{Figuras/situacaopes.png}
    \end{figure}
\end{frame}
\section{Modelos lineares}
\begin{frame}
    \tableofcontents[currentsection]
\end{frame}
\subsection{MQO}
\begin{frame}{MQO}
    \begin{itemize}
    \item \textbf{Modelo 1}:
    \begin{equation}
        log(salh) = educ +idade + idade^2  
    \end{equation} 
    \item \textbf{Modelo 2}:
    \begin{equation}
        log(salh) = educ +idade + idade^2 + fem + supc
    \end{equation} 
    \item \textbf{Modelo 3}:
    \begin{equation}
        log(salh) = educ +idade + idade^2 + fem + supc +RJ+ES+SP
    \end{equation} 
    \item \textbf{Modelo 4}:
    \begin{equation}
        log(salh) = educ +idade + idade^2 + fem + supc +rendadom+pp\\+urb+(filho6*fem)
    \end{equation}
    \end{itemize}
\end{frame}

%\begin{frame}{Retorno da educação}
%    \begin{itemize}
%        \item Modelo 1: 9,7\%
%        \item Modelo 2: 6,8\%
%        \item Modelo 3: 6,7\%
%        \item Modelo 4: 4,7\%
%        \item O retorno da educação é %estatisticamente significante no nível de 5\%
%    \end{itemize}
%\end{frame}

\begin{frame}{Modelos MQO}
    \begin{figure}
        \centering
        \includegraphics[width=.65\linewidth]{Figuras/2021-10-28 (1).png}
    \end{figure}
\end{frame}

\begin{frame}{Teste de especificação}
 \begin{table}[H]
    \caption{Teste RESET}
    \centering
    \begin{tabular}{lccccc}\hline
    \textbf{Teste}     & df1 & df2   & Estatística & P-Valor \\\hline
    RESET & 2   & 126536 & 3829.3     & <2.2e-16\\\hline
 \end{tabular}
\end{table}
\end{frame}

\subsection{MQ2E}
\begin{frame}{MQ2E}
    \item \textbf{Instrumento 1}:
    \begin{equation}
        \widehat{educ} = s1nasc + educconj + npesdom 
    \end{equation} 
    \item \textbf{Instrumento 2}:
    \begin{equation}
        \widehat{educ} = s1nasc + educconj 
    \end{equation} 
\end{frame}

\begin{frame}{Testes para as VIs}
\begin{table}[H]
    \caption{Testes dos instrumentos 1}
    \centering
    \begin{tabular}{lccccc}\hline
    \textbf{Testes}     & df1 & df2   & Estatística & P-Valor \\\hline
    Instrumentos Fracos & 2   & 43659 & 33163.2     & $<$2e-16$^{***}$\\
    Wu-Hausman          & 1   & 43660 & 208.2       & $<$2e-16$^{***}$ \\
    Sargan              & 1   & NA    & 234.6       & $<$2e-16$^{***}$ \\\hline
    \end{tabular}
\end{table}

\begin{table}[H]
    \caption{Testes dos instrumentos 2}
    \centering
    \begin{tabular}{lccccc}\hline
    \textbf{Testes}     & df1 & df2   & Estatística & P-Valor \\\hline
    Instrumentos Fracos & 2   & 43660 & 49673.267   & $<$2e-16$^{***}$\\
    Wu-Hausman          & 1   & 43660 & 191.311     & $<$2e-16$^{***}$ \\
    Sargan              & 1   & NA    & 0.827       & 0.363 \\\hline
    \end{tabular}
\end{table}
\end{frame}

\begin{frame}{MQ2E x MQO}
    \begin{figure}
        \centering
        \includegraphics[width = .55\linewidth]{Figuras/2021-10-28.png}
    \end{figure}
\end{frame}

%%%%%%%%%%%%%%%%%%%%%%%%%%%%%%%%%%%%%%%%%%%%%%%%%%%%%%%%%%%%%%%%%%%%%%%%%%%%%%%%%%%%%%%%%%%
\subsection{Painel}
\begin{frame}{Painel}
\begin{itemize}
    \item Painel desbalanceado;
    \item Indexadores: V1016 e idind;
    \item Ao mesmo tempo que o painel é grande, ele é pequeno;
    \item Estimado por diferentes métodos;
    
\end{itemize}
\end{frame}
%%%%%%%%%%%%%%%%%%%%%%%%%%%%%%%%%%%%%%%%%%%%%%%%%%%%%%%%%%%%%%%%%%%%%%%%%%%%%%%%%%%%%%%%%%%%%

\begin{frame}{Regressões}
\begin{center}
    \includegraphics[width=7cm]{Figuras/painel.png}
\end{center}
\end{frame}
%%%%%%%%%%%%%%%%%%%%%%%%%%%%%%%%%%%%%%%%%%%%%%%%%%%%%%%%%%%%%%%%%%%%%%%%%%%%%%%%%%%%%%%%%%%%%

\begin{frame}{Qual modelo escolher?}
\begin{center}
    \includegraphics[width=8cm]{Figuras/teste_painel.png}
\end{center}
\end{frame}
%%%%%%%%%%%%%%%%%%%%%%%%%%%%%%%%%%%%%%%%%%%%%%%%%%%%%%%%%%%%%%%%%%%%%%%%%%%%%%%%%%%%%%%%%%%%%
\begin{frame}{Efeitos Aleatórios Correlacionados -EAC}
\begin{itemize}
    \item \textbf{EAC 1}:
    \begin{equation}
    \begin{split}
        log(salh) = educ +idade + idade^2 +fem + supc + PP+ urb\\+ rendadom+ fem + fem*(filho6) + med\_idade+med\_idadesq\\ + med\_rendadom + med\_supc+ med\_filho6 +med\_pp\\+ med\_fem
    \end{split}
    \end{equation} 
    \item \textbf{EAC 2}:
    \begin{equation}
    \begin{split}
        log(salh) = educ +idade + idade^2 + fem + supc +PP+\\urb+ rendadom+fem + fem*(filho6)+ med_idade+\\med\_idadesq+ +med\_rendadom +med\_filho6
    \end{split}
    \end{equation}
\end{itemize}
\end{frame}
%%%%%%%%%%%%%%%%%%%%%%%%%%%%%%%%%%%%%%%%%%%%%%%%%%%%%%%%%%%%%%%%%%%%%%%%%%%%%%%%%%%%%%%%%%%%%
\begin{frame}{Efeitos Aleatórios Correlacionados -EAC}
\begin{center}
    \includegraphics[width=3cm]{Figuras/CRE.png}
\end{center}
\end{frame}
%%%%%%%%%%%%%%%%%%%%%%%%%%%%%%%%%%%%%%%%%%%%%%%%%%%%%%%%%%%%%%%%%%%%%%%%%%%%%%%%%%%%%%%%%%%%%
\section{Modelos não lineares}
\begin{frame}
    \tableofcontents[currentsection]
\end{frame}
\subsection{Modelo Logit}
\begin{frame}{Modelo Logit}
    \begin{itemize}
        \item A variável dependente de \textit{Ocupação} é binária, sendo ela igual 1 quando o indivíduo está ocupado e 0 quando ele não está.
        \item Ela viola certos pressupostos do modelo linear como o da homoscedasticidade, linearidade e normalidade.
        \item Logit ou Probit?
        \begin{itemize}
            \item A principal diferença entre elas é que no primeiro a distribuição é \textit{logística} e no outro é \textit{normal};
            \item Devido a maior praticidade de interpretação do modelo de \textit{regressão logística}, ele foi o escolhido.
        \end{itemize}
    \end{itemize}    
\end{frame}

%%%%%%%%%%%%%%%%%%%%%%%%%%%%%%%%%%%%%%%%%%%%%%%%%%%%%%%%%%%%%%%%%%%%%%%%%%%%%%%%%%%%%%%%%%%%

\begin{frame}{Critério de seleção de Akaike}
\begin{table}[H]
    \caption{Seleção de Modelo baseado em AICc}
    \centering
    \begin{tabular}{l|c|ccccc}\hline
    M. Logit& k  & AICc     & AICcWt\\\hline
    logit4  & 15 & 205407.6 & 1.00\\
    logit1  & 13 & 205430.9 & 0.00\\
    logit5  & 13 & 205430.9 & 0.00\\
    logit2  & 14 & 205758.7 & 0.00\\
    logit3  &  9 & 205758.7 & 0.00\\\hline
    \end{tabular}
\end{table}
\end{frame}

\begin{frame}{Resultados}
\begin{table}[H]
    \caption{Estimação de \textbf{logit4}}
    \centering
    \begin{adjustbox}{width=.4\columnwidth,center}
    \begin{tabular}{l|c}\hline
    Variáveis   & Coef. \\\hline
    Intercepto  & -2.511$^{***}$ \\
    Idade       & 0.25$^{***}$ \\
    Idade$^2$   & -0.003$^{***}$ \\
    Educ        & 0.013$^{***}$ \\
    SupC        & 0.421$^{***}$\\
    Fem         & -2.13$^{***}$\\
    Filho6      & 0.653$^{***}$\\
    Urb         & -0.202$^{***}$\\
    PP          & -0.145$^{***}$\\
    RendaDom    & 0.0001$^{***}$\\
    Fem:Filho6  & -1.027$^{***}$ \\
    Fem:PP      & 0.133$^{***}$ \\
    Fem:SupC    & 0.043 \\
    Educ:Urb    & 0.014$^{***}$ \\
    Educ:PP     & 0.017$^{***}$\\
    Educ:Fem    & 0.076$^{***}$\\
    \hline \\[-2ex]
    \hline
    Observações & 91.575 \\ 
    R$^2$ Ajustado & 0.460 \\
    Valor de Log{--}Verossimilhança & $-$99,187.450 \\ 
    $\hat{\lambda}$ & 0.192$^{***}$  (0.032) \\ 
    $\sigma$ & 0.576 
    \\\hline
    \end{tabular}
    \end{adjustbox}
\end{table}
\end{frame}

\begin{frame}{Resultados}
\begin{table}[H]
    \caption{Teste da Razão de Verossimilhança}
    \centering
    \begin{tabular}{l|cccc}\hline
   \textbf{ Modelos}& df & LogLik & Chisq  & Pr($>$Chisq)     \\\hline
    M. Restrito   & 13 & $-$102911 &        &                \\
    M. Irrestrito & 15 & $-$102689 & 443.45 & $<$2e-16$^{***}$ \\\hline
    \end{tabular}
\end{table}
\end{frame}

\begin{frame}{Resultados}
    \begin{table}[H]
    \caption{Efeito Parcial Médio}
    \centering
    \begin{tabular}{l|cc}\hline
    \textbf{V. Explicativas} & \textbf{logit4}  \\\hline
    Educ            & 0.0128  \\
    Fem             & -0.2451 \\
    Filho6          & -0.0083 \\
    PP              & 0.0100  \\
    SupC            & 0.0880  \\
    Idade           & -0.0013 \\
    Urb             & -0.0077 \\\hline
    \end{tabular}
\end{table}
\end{frame}
\subsection{Modelo Tobit}
\begin{frame}{Modelo Tobit: Resultados}
\begin{figure}
    \centering
    \includegraphics[width =.34\linewidth]{Figuras/2021-10-27.png}
\end{figure}
\end{frame}
\begin{frame}{Modelo Tobit: Resultados}
    \begin{table}[H]
    \centering
    \begin{tabular}{l|cc}\hline
    V. Explicativas & Tobit   & MQO     \\\hline
    Idade           & 0.0427  & 0.0118  \\
    Educ            & 0.0452  & 0.0456  \\
    PP              & -0.0628 & -0.0635 \\
    SupC            & 0.3430  & 0.3399  \\
    Fem             & -0.2512 & -0.2375 \\
    Filho6          & 0.0449  & 0.0558  \\
    Urb             & 0.1579  & 0.1618  \\
    RendaDom        & 0.00009 & 0.00009 \\\hline
    \end{tabular}
\end{table}
\end{frame}
\subsection{Modelo Poisson}
\begin{frame}{Modelo Poisson: Resultados}
\begin{table}[H]
    \caption{Seleção de Modelo baseado em AICc}
    \centering
    \begin{tabular}{l|c|ccccc}\hline
    M. Poisson& k  & AICc     & AICcWt\\\hline
    poisson3  & 14 & 122231.7 & 1.00  \\
    poisson2  & 12 & 122243.3 & 0.00  \\
    poisson4  & 13 & 122245.8 & 0.00  \\
    poisson1  & 10 & 122289.1 & 0.00  \\\hline
    \end{tabular}
\end{table}
\end{frame}

\begin{frame}{Modelo Poisson: Resultados}
    \begin{figure}
        \centering
        \includegraphics[width=.375\linewidth]{Figuras/Captura de tela 2021-10-27 202540.png}
    \end{figure}
\end{frame}

\begin{frame}{Modelo Poisson: Resultados}
    \begin{table}[H]
        \caption{Educ, Fem, PP}
        \centering
        \begin{adjustbox}{width=\columnwidth,center}
            \begin{tabular}{l|c|ccccc}\hline
    Modelos   & Res. Df & Resid. Dev & Df & Deviance & Pr($>$Chi)       \\\hline
    M. Rest.  & 43661   & 42030      &    &          & \\
    M. Irres. & 43658   & 42000      & 3  & 30.17    &  0.0000012$^{***}$ \\\hline
            \end{tabular}
        \end{adjustbox}
\end{table}
\end{frame}

\begin{frame}{Modelo Poisson: Resultados}
    \begin{figure}[H]
    \centering
    \caption{Número de filhos estimados x Idade}
    \includegraphics[width=.725\linewidth]{Figuras/00001d.png}
    \caption*{\textbf{Fonte: }PNAD Contínua - IBGE (2012)}
\end{figure}
\end{frame}
\subsection{Estimação de Heckit}
\begin{frame}{Estimação de Heckit}
    \begin{itemize}
        \item  Trabalhar pode estar correlacionado com fatores não observáveis que impactam a oferta de salários.
        \item O que pode ser um \textit{truncamento ocasional}, um tipo de seleção amostral.
        \item Uma das formas de avaliar se há de fato um problema de seleção amostral no modelo é através da Estimação de Heckit.
    \end{itemize}
\end{frame}

\begin{frame}{Resultados}
    \begin{itemize}
        \item Equação de $\widehat{lsalh}$, somente para mulheres:
    \end{itemize}
\[\widehat{lsalh} = -0.263 + 0.044Idade - 0.0004Idade^2 + 0.052Educ \\ 
    - 0.055PP + 0.45SupC + 0.00007RendaDom + 0.144Urb\]
\begin{table}[H]
\centering 
\begin{adjustbox}{width=0.6\columnwidth,center}
\begin{tabular}{lc} \hline \\[-1.8ex] 
Observações & 91.575 \\ 
R$^2$ Ajustado & 0.460 \\
Valor de Log{--}Verossimilhança & $-$99,187.450 \\ 
$\hat{\lambda}$ & 0.192$^{***}$  (0.032) \\ 
$\sigma$ & 0.576 & 
\hline 
\end{tabular} 
\end{adjustbox}
\end{table}
\end{frame}

\begin{frame}
    \titlepage
    \begin{figure}[htpb]
        \includegraphics[width=0.2\linewidth, left]{Figuras/download.png}
    \end{figure}
\end{frame}
\end{document}  
