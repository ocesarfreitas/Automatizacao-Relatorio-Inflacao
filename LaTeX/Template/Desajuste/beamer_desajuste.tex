%%%%%%%%%%%%%%%%%%%%%%%%%%%%%%%%%%%%%%%%%%%%%%%%%%%%
% Definindo o tema que irei utilizar como base
\documentclass[xcolor=dvipsnames]{beamer}
\usetheme{Copenhagen}

%%%%%%%%%%%%%%%%%%%%%%%%%% Paleta de cores 
% Aqui vou definindo a paleta de cores que irei utilizar no Beamer, utilizo o site http://latexcolor.com/ para procurar as cores
\definecolor{darkcyan}{rgb}{0.0, 0.55, 0.55}
\definecolor{coolblack}{rgb}{0.0, 0.18, 0.39}
\definecolor{midnightgreen(eaglegreen)}{rgb}{0.0, 0.29, 0.33}
\definecolor{warmblack}{rgb}{0.0, 0.26, 0.26}

% Essa parte é muito importante para "setar" como e onde serão modificadas as cores do beamer
\setbeamercolor*{structure}{bg=midnightgreen(eaglegreen)!20,fg=midnightgreen(eaglegreen)}
\setbeamercolor{palette sidebar primary}{fg=white!80!white}
\setbeamercolor*{palette primary}{use=structure,fg=white,bg=structure.fg!80}
\setbeamercolor*{palette secondary}{use=structure,fg=white,bg=structure.fg}
\setbeamercolor*{palette tertiary}{use=structure,fg=white,bg=structure.fg}
\setbeamercolor*{palette quaternary}{fg=white,bg=structure.fg!5!warmblack}
\setbeamercolor{titlelike}{parent=palette primary,fg=white}
\setbeamercolor{frametitle}{bg=gray!10!white,fg=structure.fg!75!white}
\setbeamercolor{sidebar}{fg=white,bg=white}
\setbeamercolor{navigation symbols dimmed}{fg=red!80!white}
\setbeamercolor{navigation symbols}{fg=red!80!white}
%%%%%%%%%%%%%%%%%%%%%%%%%%

%%%%%%%%%%%%%%%%%%%%%%%%%% Pacotes
\usepackage{graphicx}
\usepackage{booktabs}
\usepackage{subcaption}
\usepackage{textpos, palatino, tikz}
\usepackage{smartdiagram}
\usepackage{listings}
\usepackage{minted}
\usepackage[export]{adjustbox}
\usepackage[portuguese]{babel}
\usepackage{hyperref}
\usepackage{amsmath}
%%%%%%%%%%%%%%%%%%%%%%%%%%

%%%%%%%%%%%%%%%%%%%%%%%%%% Informações 
\lstset{language = R}

\title[Título]{Título}
\subtitle{Subtítulo}
\author[Autor 1 \and Autor 2]{Autor 2 \and autor 2}
\date{XX de XXXXXX de XXXX}
%%%%%%%%%%%%%%%%%%%%%%%%%%

%%%%%%%%%%%%%%%%%%%%%%%%%% Slides
\begin{document}
\begin{frame}
    \titlepage
    \begin{figure}[htpb]
            \includegraphics[width=.155\linewidth, left]{Figuras/logo_desajuste.png}
    \end{figure}
\end{frame}

\logo{
\includegraphics[width = .15\linewidth]{Figuras/logo_desajuste.png}
}

\begin{frame}
    \tableofcontents
\end{frame}

\section{Seção 1} % Repito o "tableofcontents" para ficar como um slide de transição entre as seções
\begin{frame}
    \tableofcontents[currentsection]
\end{frame}
\subsection{Subseção 1}
\begin{frame}{Frame Title}
    
\end{frame}
\begin{frame}[fragile]
    \frametitle{tidyverse}
    
    \rule{\textwidth}{1pt}
    \scriptsize
    \begin{minted}{R}
         # Instalando pacote
         install.package("tidyverse")
         # Importando
         library(tidyverse)
    \end{minted}
    \rule{\textwidth}{1pt}
\end{frame}

\end{document}  
%%%%%%%%%%%%%%%%%%%%%%%%%%%%%%%%%%%%%%%%%%%%%%%%%%%%
